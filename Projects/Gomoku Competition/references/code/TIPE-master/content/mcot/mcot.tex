\documentclass[12pt,a4paper]{article}
\usepackage[margin=1.5cm]{geometry}
\usepackage[utf8]{inputenc}
\usepackage[T1]{fontenc}
\usepackage[]{hyperref}
\begin{document}
\begin{center}
    {\huge \bfseries Morpion}
\end{center}
\section*{\bfseries Positionnement thématique}

\hspace*{7mm}\textit{Informatique Pratique: Intelligence artificielle.}

\section*{\bfseries Mots-clés}

\begin{tabular}{l l}
    {\bfseries Mots-clés} (en français) & {\bfseries Mots-clés} (en anglais)\\
    \textit{Morpion}& \textit{Tic-tac-toe}\\
    \textit{Intelligence artificielle} & \textit{Artificial Intelligence}\\
    \textit{Glouton} & \textit{Greedy}\\
    \textit{Minimax} & \textit{Minimax}\\
    \textit{Élagage alpha-beta} & \textit{Alpha-beta pruning}
\end{tabular}

\section*{\bfseries Bibliographie commentée}
\hspace*{7mm}Le morpion est un jeu classique dans lequel l'objectif est d'aligner 3 cercles (ou 3 croix) sur une grille de taille $3 \times 3$.~Nous intéressons ici à un jeu généralisé de taille $m \times n$.~Dans ce jeu généralisé, le premier joueur place un pion noir sur la grille, puis le second joueur y place un pion blanc et ainsi de suite.~Le but est d'aligner $k$ pions de même couleur.\newline
\hspace*{7mm}Afin d'étudier le morpion, nous considérons le jeu comme un (m, n, k)-jeu où $m\times n$ est la taille de la grille et $k$ est le nombre de pions dans une ligne pour gagner.~Avec l'argument de « voler la stratégie de l'adversaire » établie par Uiterwijk et Henrik \cite{steal}, s'il existe une stratégie gagnante pour le deuxième joueur, le premier joueur peut placer un pion et puis jouer comme s'il était le deuxième joueur.~Ainsi, soit il existe une stratégie gagnante pour le premier joueur, soit une stratégie qui dans le pire des cas mène au match nul.~Le (m, n, k)-jeu pour $k\ge 8$ est montré d'être un jeu à égalité \cite{9-win, 8-win} en divisant le jeu en plusieurs sous-jeux et en montrant que le premier joueur n'a pas de stratégies gagnantes dans tous ces sous-jeux.~Quitte à forcer le second joueur à jouer une certaine place, la méthode Threat space search, détaillé par Allis \cite{Threat}, montre qu'il existe une stratégie gagnante dans le (15, 15, 5)-jeu.~Le principe est de forcer le deuxième joueur à placer les pions en certaines places pour alléger le nombre de calculs, puis chercher tous les cas possibles.~Les (m, n, 6)-jeux et les \mbox{(m, n, 7)-jeux} restent encore indéterminés \cite{67-win}.~La théorie des jeux donne un algorithme (minimax) \cite{minimax} pour chercher une stratégie assimilée à la meilleure stratégie.~En fin, avec l'élagage alpha-beta \cite{alpha-beta}, nous pouvons améliorer l'algorithme minimax en efficacité.


\newpage

\section*{\bfseries Problématique retenue}
\hspace*{7mm}Nous ignorons encore s'il existe des stratégies gagnantes ou non pour les \mbox{(m, n, 6)-jeu} et \newline \mbox{(m, n, 7)-jeu}.~Il semble donc naturel d'étudier la stratégie de (m, n, 5)-jeu et puis l'étendre au cas de 6 et de 7. 
\section*{\bfseries Objectifs du TIPE}
\begin{enumerate}
    \item Chercher une stratégie sous la forme d'une fonction qui a l'état de la grille comme entrée et renvoie une place comme sortie, avec laquelle le premier joueur peut toujours gagner dans le (15, 15, 5)-jeu.
    \item Améliorer la stratégie avec l'algorithme de minimax puis avec l'élagage alpha-beta.
    \item Essayer de résoudre le (m, n, 6)-jeu avec notre algorithme.
\end{enumerate}
\section*{\bfseries Références bibliographiques}
\begingroup
\renewcommand{\section}[2]{}%
\begin{thebibliography}{10}
\bibitem{steal}
    Uiterwijk, J.W.H.M., Herik, H.J. van den., The advantage of the initiative, Information Sciences, 122 (2000) 43-58.
\bibitem{9-win}
    Hales, A.W., Jewett, R.I. (1963). Regularity and positional games. Transactions of the American Mathematical Society 106 222-229.
\bibitem{8-win}
    Zetters, T.G.L. Problem S.10 proposed by R.K.Guy and J.L. Selfridge, The American Mathematical Monthly 86 (1979), solution 87 575-576.
\bibitem{Threat}
    Allis, L. V. (1994) Searching for solutions in games and artificial intelligence, Ph.D. Thesis. University of Limburg, Maastricht.
\bibitem{67-win}
    \url{http://www.weijima.com/index.php?option=com_content&view=article&id=11}
\bibitem{minimax}
    Guillermo Owen, (1967) Communications to the Editor-An Elementary Proof of the Minimax Theorem. Management Science 13(9):765-765
\bibitem{alpha-beta}
    Stuart Russell, Peter Norvig. Artificial Intelligence-A Modern Approach Prentice Hall (2010) 167-171
\end{thebibliography}
\endgroup
\end{document}
